mbed T\-L\-S (formerly known as Polar\-S\-S\-L) makes it trivially easy for developers to include cryptographic and S\-S\-L/\-T\-L\-S capabilities in their embedded products, with a minimal code footprint. It offers an S\-S\-L library with an intuitive A\-P\-I and readable source code.

The Beta release of mbed T\-L\-S integrates the mbed T\-L\-S library into mbed O\-S, mbed S\-D\-K and yotta. This is a preview release intended for evaluation only and is {\bfseries not recommended for deployment}. It currently implements no secure source of random numbers, weakening its security.

\subsection*{Sample programs}

This release includes the following examples\-:
\begin{DoxyEnumerate}
\item \href{https://github.com/ARMmbed/mbedtls/blob/development/yotta/data/example-tls-client}{\tt $\ast$$\ast$\-T\-L\-S client\-:$\ast$$\ast$} found in {\ttfamily test/example-\/tls-\/client}. Downloads a test file from an H\-T\-T\-P\-S server and looks for a specific string in that file.
\item \href{https://github.com/ARMmbed/mbedtls/blob/development/yotta/data/example-selftest}{\tt $\ast$$\ast$\-Self test\-:$\ast$$\ast$} found in {\ttfamily test/example-\/selftest}. Tests different basic functions in the mbed T\-L\-S library.
\item \href{https://github.com/ARMmbed/mbedtls/blob/development/yotta/data/example-benchmark}{\tt $\ast$$\ast$\-Benchmark\-:$\ast$$\ast$} found in {\ttfamily test/example-\/benchmark}. Measures the time taken to perform basic cryptographic functions used in the library.
\end{DoxyEnumerate}

These examples are integrated as yotta tests, so that they are built automatically when you build mbed T\-L\-S. You'll find more examples in the various {\ttfamily test/example-\/$\ast$} directories.

\subsection*{Running mbed T\-L\-S}

To build and run the example, please follow the instructions in the \href{https://github.com/ARMmbed/mbedtls/blob/development/yotta/data/example-tls-client}{\tt T\-L\-S client example} directory. These include a list of prerequisites and an explanation of building mbed T\-L\-S with yotta.

\subsection*{Configuring mbed T\-L\-S features}

mbed T\-L\-S makes it easy to disable any feature during compilation that isn't required for a particular project. The default configuration enables all modern and widely-\/used features, which should meet the needs of new projects, and disables all features that are older or less common, to minimize the code footprint.

The list of available compilation flags is presented in the fully documented \href{https://github.com/ARMmbed/mbedtls/blob/development/include/mbedtls/config.h}{\tt config.\-h file}, present in the {\ttfamily mbedtls} directory of the yotta module.

If you need to adjust those flags, you can provide your own configuration file with suitable {\ttfamily \#define} and {\ttfamily \#undef} statements. These will be included between the default definitions and the sanity checks. Your configuration file should be in your application's {\ttfamily include} directory, and can be named freely; you just need to let mbed T\-L\-S know the file's name. To do that, use yotta's \href{http://docs.yottabuild.org/reference/config.html}{\tt configuration system}. The file's name should be in your {\ttfamily config.\-json} file, under mbedtls, as the key {\ttfamily user-\/config-\/file}. For example\-: \begin{DoxyVerb}{
   "mbedtls": {
      "user-config-file": "\"myapp/my_mbedtls_config_changes.h\""
   }
}
\end{DoxyVerb}


Please note\-: you need to provide the exact name that will be used in the {\ttfamily \#include} directive, including the {\ttfamily $<$$>$} or quotes around the name.

\subsection*{Contributing}

We gratefully accept bug reports and contributions from the community. There are some requirements we need to fulfill in order to be able to integrate contributions\-:
\begin{DoxyItemize}
\item Simple bug fixes to existing code do not contain copyright themselves and we can integrate without issue. The same is true of trivial contributions.
\item For larger contributions, such as a new feature, the code can possibly fall under copyright law. We then need your consent to share in the ownership of the copyright. We have a form for this, which we will send to you in case you submit a contribution or pull request that we deem this necessary for.
\end{DoxyItemize}

To contribute, please\-:
\begin{DoxyItemize}
\item \href{https://github.com/ARMmbed/mbedtls/issues}{\tt Check for open issues} or \href{https://tls.mbed.org/discussions}{\tt start a discussion} around a feature idea or a bug.
\item Fork the \href{https://github.com/ARMmbed/mbedtls}{\tt mbed T\-L\-S repository on Git\-Hub} to start making your changes. As a general rule, you should use the \char`\"{}development\char`\"{} branch as a basis.
\item Write a test that shows that the bug was fixed or that the feature works as expected.
\item Send a pull request and bug us until it gets merged and published. We will include your name in the Change\-Log \-:) 
\end{DoxyItemize}