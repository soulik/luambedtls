This application performs authenticated encryption and authenticated decryption of a buffer. It serves as a tutorial for the basic authenticated encryption functions of mbed T\-L\-S.

\subsection*{Pre-\/requisites}

To build and run this example you must have\-:
\begin{DoxyItemize}
\item A computer with the following software installed\-:
\begin{DoxyItemize}
\item \href{http://www.cmake.org/download/}{\tt C\-Make}.
\item \href{https://github.com/ARMmbed/yotta}{\tt yotta}. Please note that {\bfseries yotta has its own set of dependencies}, listed in the \href{http://armmbed.github.io/yotta/#installing-on-windows}{\tt installation instructions}.
\item \href{https://www.python.org/downloads/}{\tt Python}.
\item \href{https://launchpad.net/gcc-arm-embedded}{\tt The A\-R\-M G\-C\-C toolchain}.
\item A serial terminal emulator (Like screen, py\-Serial and cu).
\end{DoxyItemize}
\item An \href{http://developer.mbed.org/platforms/FRDM-K64F/}{\tt F\-R\-D\-M-\/\-K64\-F} development board, or another board supported by mbed O\-S (in which case you'll have to substitute frdm-\/k64f-\/gcc with the appropriate target in the instructions below).
\item A micro-\/\-U\-S\-B cable.
\item If your O\-S is Windows, please follow the installation instructions \href{https://developer.mbed.org/handbook/Windows-serial-configuration}{\tt for the serial port driver}.
\end{DoxyItemize}

\subsection*{Getting started}


\begin{DoxyEnumerate}
\item Connect the F\-R\-D\-M-\/\-K64\-F to the computer with the micro-\/\-U\-S\-B cable, being careful to use the \char`\"{}\-Open\-S\-D\-A\char`\"{} connector on the target board.
\item Navigate to the mbedtls directory supplied with your release and open a terminal.
\item Set the yotta target\-:

``` yotta target frdm-\/k64f-\/gcc ```
\item Build mbedtls and the examples. This may take a long time if this is your first compilation\-:

``` \$ yotta build ```
\item Copy {\ttfamily build/frdm-\/k64f-\/gcc/test/mbedtls-\/test-\/example-\/authcrypt.\-bin} to your mbed board and wait until the L\-E\-D next to the U\-S\-B port stops blinking.
\item Start the serial terminal emulator and connect to the virtual serial port presented by F\-R\-D\-M-\/\-K64\-F.

Use the following settings\-:
\begin{DoxyItemize}
\item 115200 baud (not 9600).
\item 8\-N1.
\item No flow control.
\end{DoxyItemize}
\item Press the Reset button on the board.
\item The output in the terminal window should look like\-:

``` \{\{timeout;10\}\} \{\{host\-\_\-test\-\_\-name;default\}\} \{\{description;mbed T\-L\-S example authcrypt\}\} \{\{test\-\_\-id;M\-B\-E\-D\-T\-L\-S\-\_\-\-E\-X\-\_\-\-A\-U\-T\-H\-C\-R\-Y\-P\-T\}\} \{\{start\}\}
\end{DoxyEnumerate}

\begin{DoxyVerb}plaintext message: 536f6d65207468696e67732061726520626574746572206c65667420756e7265616400
ciphertext: c57f7afb94f14c7977d785d08682a2596bd62ee9dcf216b8cccd997afee9b402f5de1739e8e6467aa363749ef39392e5c66622b01c7203ec0a3d14
decrypted: 536f6d65207468696e67732061726520626574746572206c65667420756e7265616400

DONE
{{success}}
{{end}}
```
\end{DoxyVerb}


The actual output for the ciphertext line will vary on each run because of the use of a random nonce in the encryption process. 