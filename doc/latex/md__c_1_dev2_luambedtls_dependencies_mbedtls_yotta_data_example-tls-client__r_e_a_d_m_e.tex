This application downloads a file from an H\-T\-T\-P\-S server (developer.\-mbed.\-org) and looks for a specific string in that file.

This example is implemented as a logic class (Hello\-H\-T\-T\-P\-S) wrapping a T\-C\-P socket and a T\-L\-S context. The logic class handles all events, leaving the main loop to just check if the process has finished.

\subsection*{Pre-\/requisites}

To build and run this example the following requirements are necessary\-:
\begin{DoxyItemize}
\item A computer with the following software installed\-:
\begin{DoxyItemize}
\item \href{http://www.cmake.org/download/}{\tt C\-Make}.
\item \href{https://github.com/ARMmbed/yotta}{\tt yotta}. Please note that {\bfseries yotta has its own set of dependencies}, listed in the \href{http://armmbed.github.io/yotta/#installing-on-windows}{\tt installation instructions}.
\item \href{https://www.python.org/downloads/}{\tt Python}.
\item \href{https://launchpad.net/gcc-arm-embedded}{\tt A\-R\-M G\-C\-C toolchain}.
\item A serial terminal emulator (e.\-g. screen, py\-Serial, cu).
\end{DoxyItemize}
\item An \href{http://developer.mbed.org/platforms/FRDM-K64F/}{\tt F\-R\-D\-M-\/\-K64\-F} development board, or another board that has an ethernet port and is supported by mbed O\-S (in which case you'll have to substitute frdm-\/k64f-\/gcc with the appropriate target in the instructions below).
\item An ethernet connection to the internet.
\item An ethernet cable.
\item A micro-\/\-U\-S\-B cable.
\item If your O\-S is Windows, please follow the installation instructions \href{https://developer.mbed.org/handbook/Windows-serial-configuration}{\tt for the serial port driver}.
\end{DoxyItemize}

\subsection*{Getting started}


\begin{DoxyEnumerate}
\item Connect the F\-R\-D\-M-\/\-K64\-F to the internet using the ethernet cable.
\item Connect the F\-R\-D\-M-\/\-K64\-F to the computer with the micro-\/\-U\-S\-B cable, being careful to use the \char`\"{}\-Open\-S\-D\-A\char`\"{} connector on the target board.
\item Navigate to the mbedtls directory supplied with your release and open a terminal.
\item Set the yotta target\-:

``` yotta target frdm-\/k64f-\/gcc ```
\item Build mbedtls and the examples. This will take a long time if it is the first time\-:

``` \$ yotta build ```
\item Copy {\ttfamily build/frdm-\/k64f-\/gcc/test/mbedtls-\/test-\/example-\/tls-\/client.\-bin} to your mbed board and wait until the L\-E\-D next to the U\-S\-B port stops blinking.
\item Start the serial terminal emulator and connect to the virtual serial port presented by F\-R\-D\-M-\/\-K64\-F. For settings, use 115200 baud, 8\-N1, no flow control. {\bfseries Warning\-:} for this example, the baud rate is not the default 9600, it is 115200.
\item Press the reset button on the board.
\item The output in the terminal window should look similar to this\-:

``` \{\{timeout;120\}\} \{\{host\-\_\-test\-\_\-name;default\}\} \{\{description;mbed T\-L\-S example H\-T\-T\-P\-S client\}\} \{\{test\-\_\-id;M\-B\-E\-D\-T\-L\-S\-\_\-\-E\-X\-\_\-\-H\-T\-T\-P\-S\-\_\-\-C\-L\-I\-E\-N\-T\}\} \{\{start\}\}

Client I\-P Address is 192.\-168.\-0.\-2 Starting D\-N\-S lookup for developer.\-mbed.\-org D\-N\-S Response Received\-: developer.\-mbed.\-org\-: 217.\-140.\-101.\-30 Connecting to 217.\-140.\-101.\-30\-:443 Connected to 217.\-140.\-101.\-30\-:443 Starting the T\-L\-S handshake... T\-L\-S connection to developer.\-mbed.\-org established Server certificate\-: cert. version \-: 3 serial number \-: 11\-:21\-:4\-E\-:4\-B\-:13\-:27\-:F0\-:89\-:21\-:F\-B\-:70\-:E\-C\-:3\-B\-:B5\-:73\-:5\-C\-:F\-F\-:B9 issuer name \-: C=B\-E, O=Global\-Sign nv-\/sa, C\-N=Global\-Sign Organization Validation C\-A -\/ S\-H\-A256 -\/ G2 subject name \-: C=G\-B, S\-T=Cambridgeshire, L=Cambridge, O=A\-R\-M Ltd, C\-N=$\ast$.mbed.\-com issued on \-: 2015-\/03-\/05 10\-:31\-:02 expires on \-: 2016-\/03-\/05 10\-:31\-:02 signed using \-: R\-S\-A with S\-H\-A-\/256 R\-S\-A key size \-: 2048 bits basic constraints \-: C\-A=false subject alt name \-: $\ast$.mbed.\-com, $\ast$.mbed.\-org, mbed.\-org, mbed.\-com key usage \-: Digital Signature, Key Encipherment ext key usage \-: T\-L\-S Web Server Authentication, T\-L\-S Web Client Authentication Certificate verification passed

H\-T\-T\-P\-S\-: Received 473 chars from server H\-T\-T\-P\-S\-: Received 200 O\-K status ... \mbox{[}O\-K\mbox{]} H\-T\-T\-P\-S\-: Received 'Hello world!' status ... \mbox{[}O\-K\mbox{]} H\-T\-T\-P\-S\-: Received message\-:

H\-T\-T\-P/1.\-1 200 O\-K Server\-: nginx/1.\-7.\-10 Date\-: Tue, 18 Aug 2015 18\-:34\-:04 G\-M\-T Content-\/\-Type\-: text/plain Content-\/\-Length\-: 14 Connection\-: keep-\/alive Last-\/\-Modified\-: Fri, 27 Jul 2012 13\-:30\-:34 G\-M\-T Accept-\/\-Ranges\-: bytes Cache-\/\-Control\-: max-\/age=36000 Expires\-: Wed, 19 Aug 2015 04\-:34\-:04 G\-M\-T X-\/\-Upstream-\/\-L3\-: 172.\-17.\-42.\-1\-:8080 X-\/\-Upstream-\/\-L2\-: developer-\/sjc-\/indigo-\/2-\/nginx X-\/\-Upstream-\/\-L1-\/next-\/hop\-: 217.\-140.\-101.\-86\-:8001 X-\/\-Upstream-\/\-L1\-: developer-\/sjc-\/indigo-\/border-\/nginx

Hello world! \{\{success\}\} \{\{end\}\} ```
\end{DoxyEnumerate}

\subsection*{Debugging the T\-L\-S connection}

If you are experiencing problems with this example, you should first rule out network issues by making sure the \href{https://github.com/ARMmbed/mbed-example-network-private/tree/maste r/test/helloworld-tcpclient}{\tt simple H\-T\-T\-P file downloader example} for the T\-C\-P module works as expected. If not, please follow the debug instructions for this example.

To print out more debug information about the T\-L\-S connection, edit the file {\ttfamily source/main.\-cpp} and change the definition of {\ttfamily D\-E\-B\-U\-G\-\_\-\-L\-E\-V\-E\-L} near the top of the file from 0 to a positive number\-:
\begin{DoxyItemize}
\item Level 1 only prints non-\/zero return codes from S\-S\-L functions and information about the full certificate chain being verified.
\item Level 2 prints more information about internal state updates.
\item Level 3 is intermediate.
\item Level 4 (the maximum) includes full binary dumps of the packets.
\end{DoxyItemize}

If the T\-L\-S connection is failing with an error similar to\-: \begin{DoxyVerb}```
mbedtls_ssl_write() failed: -0x2700 (-9984): X509 - Certificate verification failed, e.g. CRL, CA or signature check failed
Failed to fetch /media/uploads/mbed_official/hello.txt from developer.mbed.org:443
```
\end{DoxyVerb}


it probably means you need to update the contents of the {\ttfamily S\-S\-L\-\_\-\-C\-A\-\_\-\-P\-E\-M} constant (this can happen if you modify {\ttfamily H\-T\-T\-P\-S\-\_\-\-S\-E\-R\-V\-E\-R\-\_\-\-N\-A\-M\-E}, or when {\ttfamily developer.\-mbed.\-org} switches to a new C\-A when updating its certificate). Another reason for this error may be a proxy providing a different certificate. Proxys can be used in some network configurations or for performing man-\/in-\/the-\/middle attacks. If you choose to ignore this error and proceed with the connection anyway, you can change the definition of {\ttfamily U\-N\-S\-A\-F\-E} near the top of the file from 0 to 1. {\bfseries Warning\-:} this removes all security against a possible attacker, therefore use at your own risk, or for debugging only! 